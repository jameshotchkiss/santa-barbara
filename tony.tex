
%%%%%%%%%%%%%%%%%%%%%%%%%%%

%Document class
\documentclass[11pt, reqno]{amsart}

%Margins and spacing
\usepackage[top=3.75cm, bottom=3cm, left=3cm, right=3cm]{geometry}
\frenchspacing    

%Packages
\usepackage{mathscinet}
\usepackage{amsmath}
\usepackage{amsthm}
\usepackage{amsfonts}
\usepackage{amssymb}
\usepackage{mathtools}
%\usepackage{eucal}
\usepackage{bm}
\usepackage[colorlinks=true, citecolor=citation, urlcolor=citation, linkcolor=reference, linktocpage = true]{hyperref}
%linktocpage=true
\usepackage{color}
\usepackage[all,cmtip]{xy}
\usepackage{enumitem}
\usepackage{tikz}  
\usepackage{graphicx}
\usepackage{scalerel}
%\usepackage{todonotes}
\usepackage{comment}
\usepackage[mathscr]{eucal}
\usepackage{stmaryrd}
\usepackage{appendix}
\usepackage{indentfirst}

%added by James
\usepackage{tikz-cd}
\usetikzlibrary{decorations.pathmorphing}

%Color definitions 
\definecolor{citation}{rgb}{0,.40,.80}
\definecolor{reference}{rgb}{.80,0,.40}

%Numbering
\numberwithin{equation}{section}

%Theorem environments
\theoremstyle{plain}
\newtheorem{theorem}{Theorem}[section]
\newtheorem{lemma}[theorem]{Lemma}
\newtheorem{proposition}[theorem]{Proposition}
\newtheorem{corollary}[theorem]{Corollary}
\newtheorem{conjecture}[theorem]{Conjecture}
\newtheorem{question}[theorem]{Question}

\theoremstyle{definition}
\newtheorem{definition}[theorem]{Definition}
\newtheorem{example}[theorem]{Example}
\newtheorem{remark}[theorem]{Remark}
\newtheorem{notation}[theorem]{Notation}
\newtheorem{warning}[theorem]{Warning}
\newtheorem{construction}[theorem]{Construction}
\newtheorem{convention}[theorem]{Convention}

\newtheorem{reduction}[theorem]{Reduction}
\newcommand{\Step}[1] {\medskip \noindent {\em Step #1.\/}}
\newtheorem{situation}[theorem]{Situation}

\makeatletter
\newtheoremstyle{italicsname}% <name>
 {3pt}% <Space above>
 {3pt}% <Space below>
 {\itshape}% <Body font>
 {}% <Indent amount>
{\bf}% <Theorem head font>
% {\itshape}% <Theorem head font>
 {.}% <Punctuation after theorem head>
 {.5em}% <Space after theorem heading>
 {\thmname{#1}\thmnumber{\@ifnotempty{#1}{ }#2}%
 \thmnote{ {\the\thm@notefont(#3)}}}% <Theorem head spec (can be left empty, meaning `normal')>
\makeatother
\theoremstyle{italicsname}
\newtheorem{innerstep}{Step}
\newenvironment{step}[1]
 {\renewcommand\theinnerstep{#1}\innerstep}
 {\endinnerstep}


%List indentation
\setlist[itemize]{topsep=5pt,itemsep=3pt}
\setlist[enumerate]{topsep=5pt,itemsep=3pt}
%{leftmargin=*, itemsep={2pt}} 


%Set notation 
\newcommand{\st}{\mid} 
\newcommand{\set}[1]{\left\{ \, #1 \, \right\}}

%Categories, functors
\newcommand{\Perf}{\mathrm{Perf}}
\newcommand{\Db}{\mathrm{D^b}}
\newcommand{\Dperf}{\mathrm{D}_{\mathrm{perf}}}
\newcommand{\Dqc}{\mathrm{D}_{\mathrm{qc}}}
\newcommand{\llangle}{\left \langle}
\newcommand{\rrangle}{\right \rangle}
\DeclareMathOperator{\Forg}{Forg}


\DeclareMathOperator{\Hdg}{Hdg}


\DeclareMathOperator{\colim}{colim}

\newcommand{\cl}{{\mathrm{cl}}}

\newcommand{\Mod}{\mathrm{Mod}}

\newcommand{\Cl}{{\mathop{\mathcal{C}\!\ell}}}
\newcommand{\cB}{\mathcal{B}}

\newcommand{\Fun}{\mathrm{Fun}}
\newcommand{\Ind}{\mathrm{Ind}}

\newcommand{\Equiv}{\mathcal{E}\!{\mathit quiv}}

\newcommand{\Aff}{\mathrm{Aff}}
\newcommand{\Sets}{\mathrm{Sets}}

\newcommand{\Nm}{\mathrm{Nm}}

\newcommand{\Sch}{\mathrm{Sch}}
\newcommand{\dSch}{\mathrm{dSch}}
\newcommand{\Grpd}{\mathrm{Grpd}}

\newcommand{\dSt}{\mathrm{dStk}}
\newcommand{\St}{\mathrm{Stk}}
\newcommand{\dArtSt}{\mathrm{dArtSt}}
\newcommand{\one}{\mathbf{1}}

\newcommand{\FM}{\mathrm{FM}}


%Varieties and maps 
\newcommand{\Gr}{\mathrm{Gr}}
\DeclareMathOperator{\Sym}{Sym}
\DeclareMathOperator{\Proj}{Proj}
\DeclareMathOperator{\Pic}{Pic}
\DeclareMathOperator{\cPic}{\mathcal{P}{\it ic}}

\newcommand{\tX}{\widetilde{X}}
\newcommand{\wtilde}{\widetilde}


\newcommand{\tM}{\widetilde{M}}

\DeclareMathOperator{\Spec}{Spec}

\newcommand{\pug}{\mathrm{pug}}
\newcommand{\op}{\mathrm{op}}
\newcommand{\br}{\mathrm{br}}

\newcommand{\dRing}{\mathrm{dRing}}
\newcommand{\dAff}{\mathrm{dAff}}

\DeclareMathOperator{\coev}{coev}
\DeclareMathOperator{\eva}{ev}

\DeclareMathOperator{\Tr}{Tr}
\DeclareMathOperator{\Sp}{Sp}


%Hom, etc.
\newcommand{\cHom}{\mathcal{H}\!{\it om}}
\DeclareMathOperator{\Hom}{Hom}
\DeclareMathOperator{\End}{End}
\DeclareMathOperator{\Ext}{Ext}
\DeclareMathOperator{\Aut}{Aut}
\DeclareMathOperator{\cAut}{\mathcal{A}{\it ut}}
\DeclareMathOperator{\ucAut}{\underline{\cAut}}
\DeclareMathOperator{\uAut}{\underline{\Aut}}
\DeclareMathOperator{\fAut}{\mathfrak{A}{\it ut}}
\DeclareMathOperator{\ufAut}{\underline{\fAut}}
\DeclareMathOperator{\BAut}{BAut}
\DeclareMathOperator{\HH}{HH}
\DeclareMathOperator{\cHH}{\mathcal{HH}}
\DeclareMathOperator{\Map}{Map}

\newcommand{\re}{\mathrm{e}}
\newcommand{\ph}{\mathrm{ph}}

%Random

\DeclareMathOperator\coker{coker}

\newcommand{\svee}{\scriptscriptstyle\vee}
\newcommand{\id}{\mathrm{id}}
\newcommand{\pr}{\mathrm{pr}}

\newcommand{\ob}{\mathrm{ob}}
\newcommand{\ind}{\mathrm{ind}}

\newcommand{\der}{\mathrm{d}}
\newcommand{\gl}{\mathrm{gl}}

\newcommand{\act}{\mathrm{act}}

\newcommand{\cUv}{\mathcal{U}^{\vee}}
\newcommand{\hcS}{\widehat{\mathcal{S}}}
\newcommand{\cSv}{\cS^{\vee}}
\newcommand{\tcK}{\wtilde{\mathcal{K}}}
\newcommand{\vV}{V^{\vee}}

\newcommand{\Coh}{\mathrm{Coh}}
\newcommand{\Cat}{\mathrm{Cat}}
\newcommand{\bB}{\mathbf{B}}
\newcommand{\PrCat}{\mathrm{PrCat}}
\newcommand{\Catcl}{\mathrm{Cat}^{\mathrm{cl}}}


\newcommand{\Ku}{\mathcal{K}u}

\DeclareMathOperator{\characteristic}{char}

\newcommand{\num}{\mathrm{num}}
\DeclareMathOperator{\Knum}{\rK_{\num}}

\DeclareMathOperator{\Stab}{Stab}

\newcommand{\tH}{\widetilde{\rH}}

\DeclareMathOperator{\Br}{Br}
\DeclareMathOperator{\BrAz}{Br_{Az}}
\DeclareMathOperator{\Div}{Div}

\DeclareMathOperator{\CH}{CH}

\newcommand{\prim}{\mathrm{prim}}

\newcommand{\usigma}{\underline{\sigma}}

\newcommand{\dP}{\mathrm{dP}}

\newcommand{\rtop}{\mathrm{top}}
\newcommand{\disc}{\mathrm{disc}}

\newcommand{\ord}{\mathrm{ord}}
\newcommand{\an}{\mathrm{an}}

\newcommand{\Ktop}[1][]{\rK_{#1}^{\rtop}}

\DeclareMathOperator{\HP}{HP}

\newcommand{\CG}{\mathrm{CG}}

\newcommand{\vir}{\mathrm{vir}}

\newcommand{\per}{\mathrm{per}}

\DeclareMathOperator{\DT}{DT}
\DeclareMathOperator{\Pf}{Pf}
\DeclareMathOperator{\amp}{amp}
\newcommand{\class}{u}
\newcommand{\init}{1}
\newcommand{\fin}{0}

\def\blank{-}


\DeclareMathOperator{\NS}{NS}
% \DeclareMathOperator{\per}{per}
\newcommand{\bF}{\mathbf{F}}
\DeclareMathOperator{\Spin}{Spin}
\newcommand{\stdim}{\mathrm{st.dim}}
\DeclareMathOperator{\Supp}{Supp}
\newcommand{\rv}{\mathrm{v}}
\newcommand{\bv}{\mathbf{v}}
\newcommand{\bw}{\mathbf{w}}
\newcommand{\et}{\mathrm{\acute{e}t}}
\newcommand{\tors}{\mathrm{tors}}
\newcommand{\pt}{\mathrm{pt}}
\newcommand{\muk}{\widetilde{\rH}}
\newcommand{\topo}{\mathrm{top}}
\newcommand{\Cliff}{\mathrm{Cl}}
\newcommand{\fso}{\mathfrak{so}}
\newcommand{\fgl}{\mathfrak{gl}}
\newcommand{\triv}{\mathrm{triv}}
\newcommand{\medwedge}{\textstyle{\bigwedge}}
\newcommand{\ev}{\mathrm{ev}}
\newcommand{\length}{\ell}
\newcommand{\KU}{\mathrm{KU}}
\newcommand{\gr}{\mathrm{gr}}
\newcommand{\Atom}{\mathrm{Atom}}

%added in November
\newcommand{\bmu}{\bm{\mu}}
\newcommand{\SL}{\mathrm{SL}}
\newcommand{\PGL}{\mathrm{PGL}}

\DeclareMathOperator{\cone}{cone}
\DeclareMathOperator{\cofib}{cofib}
\DeclareMathOperator{\fib}{fib}


%Chern characters
\DeclareMathOperator{\ch}{{ch}}
\DeclareMathOperator{\rk}{{rk}}
\newcommand{\td}{\mathrm{td}}

%Calligraphic font
\newcommand{\cO}{\mathcal{O}}
\newcommand{\cA}{\mathcal{A}}
\newcommand{\cC}{\mathscr{C}}
%\newcommand{\cD}{\mathcal{D}}
\newcommand{\cD}{\mathscr{D}}
\newcommand{\cE}{\mathcal{E}}
\newcommand{\cF}{\mathcal{F}}
\newcommand{\cG}{\mathcal{G}}
\newcommand{\cH}{\mathcal{H}}
\newcommand{\cI}{\mathscr{I}}
\newcommand{\cJ}{\mathcal{J}}
\newcommand{\cK}{\mathcal{K}}
\newcommand{\cL}{\mathcal{L}}
\newcommand{\cM}{\mathcal{M}}
\newcommand{\cN}{\mathcal{N}}
\newcommand{\cP}{\mathcal{P}}
\newcommand{\cQ}{\mathcal{Q}}
\newcommand{\cR}{\mathcal{R}}
\newcommand{\cS}{\mathcal{S}}
\newcommand{\cT}{\mathcal{T}}
\newcommand{\cU}{\mathcal{U}}
\newcommand{\cV}{\mathcal{V}}
\newcommand{\cW}{\mathcal{W}}
\newcommand{\cX}{\mathcal{X}}
\newcommand{\cY}{\mathcal{Y}}
\newcommand{\cZ}{\mathcal{Z}}

%Alternate calligraphic font 
\newcommand{\ccA}{\mathscr{A}}
\newcommand{\ccE}{\mathscr{E}}
\newcommand{\ccK}{\mathcal{K}}
\newcommand{\ccX}{\mathcal{X}}
\newcommand{\ccZ}{\mathcal{Z}}
\newcommand{\ccP}{\mathcal{P}}

%Regular font
\newcommand{\rA}{\mathrm{A}}
\newcommand{\rB}{\mathrm{B}}
\newcommand{\rC}{\mathrm{C}}
\newcommand{\rD}{\mathrm{D}}
\newcommand{\rE}{\mathrm{E}}
\newcommand{\rF}{\mathrm{F}}
\newcommand{\rG}{\mathrm{G}}
\newcommand{\rg}{\mathrm{g}}
\newcommand{\rH}{\mathrm{H}}
\newcommand{\rh}{\mathrm{h}}
\newcommand{\rK}{\mathrm{K}}
\newcommand{\rS}{\mathrm{S}}
\newcommand{\rL}{\mathrm{L}}
\newcommand{\rM}{\mathrm{M}}
\newcommand{\rN}{\mathrm{N}}
\newcommand{\rR}{\mathrm{R}}
\newcommand{\rO}{\mathrm{O}}
\newcommand{\rP}{\mathrm{P}}
\newcommand{\rQ}{\mathrm{Q}}
\newcommand{\rT}{\mathrm{T}}
\newcommand{\rU}{\mathrm{U}}
\newcommand{\rV}{\mathrm{V}}
\newcommand{\rZ}{\mathrm{Z}}

%Fraktur font 
\newcommand{\fa}{\mathfrak{a}}
\newcommand{\fj}{\mathbf{j}}
\newcommand{\fq}{\mathfrak{q}}
\newcommand{\fp}{\mathfrak{p}}
\newcommand{\fG}{\mathfrak{G}}
\newcommand{\fH}{\mathfrak{H}}
\newcommand{\fm}{\mathfrak{m}}
\newcommand{\fM}{\mathfrak{M}}
\newcommand{\fX}{\mathfrak{X}}
\newcommand{\fY}{\mathfrak{Y}}


%Bold font
\newcommand{\bA}{\mathbf{A}}
\newcommand{\bC}{\mathbf{C}}
\newcommand{\bD}{\mathbf{D}}
\newcommand{\bE}{\mathbf{E}}
\newcommand{\bG}{\mathbf{G}}
\newcommand{\bH}{\mathbf{H}}
\newcommand{\bJ}{\mathbf{J}}
\newcommand{\bZ}{\mathbf{Z}}
\newcommand{\bP}{\mathbf{P}}
\newcommand{\bQ}{\mathbf{Q}}
\newcommand{\bR}{\mathbf{R}}
\newcommand{\bY}{\mathbf{Y}}
\newcommand{\bS}{\mathbf{S}}
\newcommand{\bk}{\mathbf{k}}
\newcommand{\bj}{\mathbf{j}}
\newcommand{\bV}{\mathbf{V}}

%Sans serif font
\newcommand{\sO}{\mathsf{O}}
\newcommand{\sR}{\mathsf{R}}
\newcommand{\sA}{\mathsf{A}}
\newcommand{\sY}{\mathsf{Y}}
\newcommand{\sZ}{\mathsf{Z}}

\newcommand{\vdim}{\mathrm{vdim}}
\newcommand{\Spf}{\mathrm{Spf}}
\newcommand{\Crit}{\mathrm{Crit}}
\newcommand{\euler}{\mathrm{Eu}}

%%%%%%%%%%%%%%%%%%%%%%%%%%%%%%

%%%%%% Table of contents formatting %%%%%% 

\usepackage{xpatch}
\makeatletter   
\xpatchcmd{\@tocline}
{\hfil\hbox to\@pnumwidth{\@tocpagenum{#7}}\par}
{\ifnum#1<0\hfill\else\dotfill\fi\hbox to\@pnumwidth{\@tocpagenum{#7}}\par}
{}{}
\makeatother 


%%%%%% Redefining \part %%%%%%

\makeatletter
%default definition of article.cls
%using \renewcommand instead of \newcommand
\renewcommand\part{%
   \if@noskipsec \leavevmode \fi
   \par
   \addvspace{4ex}%
   \@afterindentfalse
   \secdef\@part\@spart}

\def\@part[#1]#2{%
    \ifnum \c@secnumdepth >\m@ne
      \refstepcounter{part}%
      \addcontentsline{toc}{part}{Part \thepart.\hspace{1em}#1}%
%      \addcontentsline{toc}{part}{\thepart.\hspace{1em}#1}%
    \else
      \addcontentsline{toc}{part}{#1}%
    \fi
    {\parindent \z@ \raggedright
     \interlinepenalty \@M
     \normalfont
     \ifnum \c@secnumdepth >\m@ne
     \centering 
     \Large\bfseries \partname\nobreakspace\thepart     
       \nobreak. 
     \fi
     \Large \bfseries { #2}%
     %%%\markboth{}{}\par}% removing redefinition of headings
     \par}%
    \nobreak
    \vskip 3ex
    \@afterheading}
\def\@spart#1{%
    {\parindent \z@ \raggedright
     \interlinepenalty \@M
     \normalfont
     \huge \bfseries #1\par}%
     \nobreak
     \vskip 3ex
     \@afterheading}
\makeatother

\renewcommand{\thepart}{\Roman{part}}


%%%%%%%%%%%%%%%%%%%%%%%%%%%%%%

% Editing commands 
\newcommand{\alex}[1]{\leavevmode{\color{blue}{#1}}}
\newcommand{\james}[1]{\leavevmode{\color{magenta}{#1}}}

\newcommand{\todo}[1]{\vspace{5 mm}\par \noindent
\marginpar{\hfill \textsc{ToDo}}
\framebox{\begin{minipage}[c]{0.95 \textwidth}
 #1 \end{minipage}}\vspace{5 mm}\par}
 \newcommand{\done}[1]{\vspace{5 mm}\par \noindent
\marginpar{\textsc{DONE!}}
\framebox{\begin{minipage}[c]{0.95 \textwidth}
 #1 \end{minipage}}\vspace{5 mm}\par}

%%%%%%%%%%%%%%%%%%%%%%%%%%%%%%

%%%%%%%%%%%%%%%%%%%%%%%%%%%%%%%%%%%%%%%%%%%%%%%%%%%%%%


\overfullrule=1mm

%%%%%%%%%%%%%%%%%%%%%%%%%%%%%%%%%%%%%%%%%%%%%%%%%%%%%%
%%%%%%%%%%%%%%%%%%%%%%%%%%%%%%%%%%%%%%%%%%%%%%%%%%%%%%


\begin{document}

\title{F-bundles and blowups}

\author{Notes on a series of talks by Tony Yue Yu \\
James Hotchkiss}

%\thanks{A.P. was partially supported by NSF grants DMS-2112747, DMS-2052750, and DMS-2143271, and a Sloan Research Fellowship.} 

%\date{\today}

\begin{abstract}
    From the abstract: ``I will discuss the basic ideas and properties of F-bundles and non-commutative Hodge structures, as well as applications to birational geometry. Joint work with Katzarkov, Kontsevich and Pantev.'' 
\end{abstract}

\maketitle

\setcounter{tocdepth}{1}
\tableofcontents

\section*{Introduction} % (fold)
\label{sec:intro}



Outline:
\begin{enumerate} [label = (\arabic*)]
    \item Review of Gromov--Witten theory
    \item nc-Hodge structures and mirror symmetry
    \item F-bundles
    \item Blowup decomposition
    \item Atoms
    \item Framing and uniqueness
\end{enumerate}

% section outline (end)




\section{Review of Gromov--Witten theory} % (fold)
\label{sec:review_of_gromov_witten_theory}

For the time being, we work with smooth, projective varieties over $\bC$.

\begin{example}
    Let $N_d$ be the number of rational curves in $\bP^2$ of degree $d$, passing through $3d - 1$ points in general position. 
    \[
        N_d = 1, 1, 12, 620, 87304, 26312976, 14616808192, \dots
    \]
    Some of the numbers are classical, but it turns out that in general, one may compute them using Kontsevich's recursive formula:
    \[
        N_d = \sum_{d_1 + d_2} N_{d_1} N_{d_2} d_1^2d_2 \left(\binom{3d - 4}{3d_1 - 2} - d_1 \binom{3d - 4}{3d_1 - 1} \right) .
    \]
\end{example}
To set up any curve-counting theory, there are two issues to take care of:
\begin{enumerate} [label = (\arabic*)]
    \item Compactness
    \item Transversality
\end{enumerate}
Different strategies for solving these issues lead to different curve-counting theories. We will use the oldest theory, Gromov--Witten theory, which also happens to be the most general (e.g., it works for varieties of arbitrary dimension).


\subsection*{Gromov--Witten theory} 
Consider maps
\[
    f:(C, p_1, \dots, p_n) \to X,
\]
where $(C, p_1, \dots, p_n)$ is an $n$-pointed proper nodal curve, and the automorphism group $\Aut(f)$ is required to be finite. Such a map is called a \emph{stable map}.

The moduli stack $\bar{\cM} = \bar{\cM}_{g, n}(X, \beta)$ of genus $g$ stable maps to $X$, where $f_* [C] = \beta \in \rH_2(X, \bZ)$. The expected dimension of $\bar{\cM}$ is 
\[
    \vdim(\bar{\cM}) = (1 - g)(\dim X - 3) + \int_{\beta} c_1(T_X) + n.
\]
Then $\bar{\cM}$ carries a virtual fundamental class $[\bar \cM]^{\vir}$ in $\CH_{\vdim}(\bar{\cM})$.

\subsection*{Rational Gromov--Witten invariants}
Let $\phi_i \in \rH^2(X, \bQ)$. Define 
\[
    \langle \phi_1, \dots, \phi_n \rangle_\beta = \int_{[\bar \cM]^\vir} \prod_{i = 1}^n \ev_i^* \phi_i \in \bQ,
\]
where $\ev_i:\bar\cM \to X$ is the evaluation at $p_i$. 
These numbers satisfy a number of interesting properties:
\begin{enumerate} [label = (\arabic*)]
    \item Symmetry: $\langle \phi_{\sigma_1}, \dots, \phi_{\sigma_n} \rangle_{\beta} = (-1)^{\sigma} \langle \phi_1, \dots, \phi_n \rangle_\beta$.
    \item Support: $\langle \phi_1, \dots, \phi_n \rangle_\beta \neq 0$ implies that $\beta = 0$ or $\beta$ is an effective curve class.
    \item Constant maps: $\langle \phi_1, \dots, \phi_n \rangle_0 \neq 0$ implies that $n = 3$, and
    \[
         \langle \phi_1, \phi_2, \phi_3 \rangle_0 = \int_{[X]} \phi_1 \cup \phi_2 \cup \phi_2.
    \]
    \item Dimension: $\langle \phi_1, \dots, \phi_n \rangle_\beta \neq 0$ implies $\sum \deg \phi_i = 2 \vdim$.
    \item Unit: $\langle 1_X, \phi_2 \dots, \phi_n \rangle_\beta  =0$ if $\beta \neq 0$.
    \item Divisor: $\langle \phi_1, \dots, \phi_n, \phi_{n + 1} \rangle_\beta = \langle \phi_1, \dots, \phi_n \rangle_\beta \cdot \int_{\beta} \phi_{n + 1}$ if $\deg \phi_{n + 1}$ and $n \geq 3$.
    \item WDVV: 
    \begin{equation*}
        \sum_{\beta' + \beta'' = \beta} \quad  \sum_{I' \cup I'' = \{5, \dots, n\}} \sum_i \langle \phi_1,\phi_2,\phi_{I'}, T_i \rangle_{\beta'} \cdot \langle \phi_5,\phi_4,\phi_{I''}, T^i \rangle_{\beta''} 
    \end{equation*}
    is symmetry under swapping $\phi_2$ and $\phi_3$. Here, $T_i$ and $T^i$ are dual bases of $\rH^*(X, \bQ)$.
\end{enumerate}

\subsection*{Gromov--Witten potential} 
Choose a basis $T_i$ of $\rH^*(X, \bQ)$. The \emph{Gromov--Witten potential} is given by
\[
    \Phi = \sum_{n \geq 0, \beta} \frac{q^{\beta}}{n!} \sum_{i_1, \dots, i_n} \langle T_{i_1}, \dots, T_{i_n} \rangle_{\beta} \in \bQ[[\mathrm{NE}(X, \bZ)]][[\{t_i\}]].
\]
We denote the coefficient ring by $R$. 
The potential leads to the quantum product
\[
    H \otimes H \to H \otimes R, \quad T_i \star T_j := \sum_k \frac{\partial^3 \Phi}{\partial t_i \partial t_j \partial k} T^k.
\]
Then WDVV implies that $\star$ is associative. 
In fact, WDVV is equivalent to the associativity of the quantum product.

\begin{conjecture}
\label{conj:convergence}
    The series $\Phi$ is convergent for small $|q|, |t_i|$.
\end{conjecture}

% section review_of_gromov_witten_theory (end)



\section{nc-Hodge structures and mirror symmetry} % (fold)
\label{sec:nc_hodge_structures_and_mirror_symmetry}

\subsection*{nc-Hodge structures} We package Gromov--Witten theory further, with the goal of bringing in ideas from mirror symmetry. A reference for some of the following material is \cite{MR2483750}.

\begin{definition}
\label{def:nc_hodge}
    A \emph{rational pure nc-Hodge structure} consists of a tuple $(H, E_B, \mathrm{iso})$, where:
    \begin{itemize}
        \item $H$ is a $\bZ/2$-graded algebraic vector bundle on $\bA^1$ with coordinate $u$,
        \item $E_B$ is a local system of finite-dimensional $\bQ$-vector spaces on $\bA^1 - 0$,
        \item $\mathrm{iso}$ is an isomorphism
        \[
            E_B \simeq \cO_{\bA^1 - 0}^{\an} \simeq H^{\an}|_{\bA^1 - 0},
        \]
        so that $H^{\an}$ inherits a flat holomorphic connection $\nabla$ on $\bA^1 - 0$.
    \end{itemize}
    such that the following conditions hold:
    \begin{enumerate} [label = (\arabic*)]
        \item nc-filtration axiom: $\nabla$ is meromorphic on $H$ with a pole of order $\leq 2$ at $ u = 0$, and a regular singularity at $\infty$.
        \item $\bQ$-structure axiom: $\cE_B$ is compatible with Stokes data. 
        \item Opposedness axiom (omitted).
    \end{enumerate}
\end{definition}

\begin{example}[Classical Hodge structures ]
    Let $(V, \rF^\bullet V, V_{\bQ})$ be a pure $\bQ$-Hodge structure of weight $w$. Then the corresponding nc-Hodge structure is given as follows:
    \begin{itemize}
        \item $H = \sum u^{-1} \rF^i V[u]$, considered as a submodule of $\bC[u,u^{-1}] \otimes V$.
        \item The connection is given by 
        \[
            \nabla = d - \frac{w}{2} \frac{du}{u}.
        \]
        \item The local system is the extension of $V_{\bQ}$, which we think of as a vector subspace of the fiber of $H$ at $1 \in \bA^1$, to a local system in $H$ by parallel transport.
    \end{itemize}
\end{example}

In general, meromorphic connections with poles of order $2$ are difficult, so we focus instead on a class where the pole is reasonable:

\begin{definition}
\label{def:exponential-type}
    An nc-Hodge structure is of \emph{exponential type} with exponents $c_1, \dots, c_n$ if, after tensoring with $\bC((u))$, it splits into a direct sum  
    \[
        \bigoplus (\cE^{c_i/u} \otimes R_i),
    \]
    where $\cE^{c_i/u}$ is a rank one vector bundle with connection $d - d\left( \frac{c_i}{u} \right)$, and $R_i$ has regular singularities. 
\end{definition}


\subsection*{Variation of nc-Hodge structures}
There is a notion of a variation of nc-Hodge structure. We omit the precise definition. Suffice it to say that the definition consists of the following ingredients:
\begin{itemize}
    \item A $\bZ/2$-graded vector bundle on $\bA^1 \times B$, which is algebraic in the $\bA^1$ direction.
    \item A local system $E_B$ of $\bZ/2$-graded vector spaces on $\bG_m \times B$.
    \item An isomorphism 
    \[
        \mathrm{iso}:E_B \otimes \cO_{\bG_m \times S} \simeq H_{\bG_m \times S},
    \]
\end{itemize}
satisfying several conditions:
\begin{enumerate} [label = (\arabic*)]
    \item nc-filtration axiom
    \item Griffiths transversality
    \item $\bQ$-structure axiom
    \item Opposedness 
\end{enumerate}


\subsection*{A-model nc-Hodge structure}

Let $B = \Spf(\bQ[[\mathrm{NE}(X, \bZ)]][[\{t_i\}]])$, which serves as the base of the variation.
Let $H$ be the trivial bundle over $B \times \bA^1_u$ with fiber $\rH= \rH^*(X, \bQ)$. We take $\nabla$ to be the \emph{quantum connection}:
\begin{equation*}
    \nabla = \begin{cases}
        \nabla_{\partial_u} = \partial_u + \frac{1}{u^2} K + \frac{1}{u} G \\
        \nabla_{\partial_{t_i}} = \partial_{t_i} + \frac{1}{u} A_i \\
        \nabla_{q_j \partial_{q_j}} = q_j \partial_{q_j} + \frac{1}{u} A_j
    \end{cases}
\end{equation*}
Here, when we write $q_j$, we mean the variable $q^\beta$ associated that an effective curve class in a chosen basis. 
The quantum operators are:
\begin{align*}
    K &= \left(K_X + \sum_{i:\deg T_i \neq 2} \frac{\deg T_i - 2}{2} t_i T_i \right) \star (-) \\
    G &= \sum_{i = 0}^{\dim X} \frac{i - \dim X}{2} \id_{\rH^i(X, \bQ)} \\
    A_i &= T_i \star (-).
\end{align*}
These operators preserve the $\bZ/2$-grading.

\begin{conjecture}
    $(H, \nabla)$ has exponential type. 
\end{conjecture}

Let us assume convergence (Conjecture~\ref{conj:convergence}), and explain some of the other aspects of the (conjectural) variation. The $\bQ$-structure is induced by the image 
\[
    \begin{tikzcd}
        \rH^k(X, \bQ) \ar[r, "(2 \pi i)^{k/2}"] & \rH^k(X, \bC) \ar[r, "\hat \Gamma(X) \wedge -"] & \rH^k(X, \bC),
    \end{tikzcd}
\]
where 
\begin{align*}
    \hat\Gamma(X) &= \prod \Gamma(1 + \lambda_i) \\
        &= \exp\left(\gamma \cdot \ch_i(T_X)) + \sum_{n \geq 2} \frac{\zeta(n)}{n} \ch_n(T_X) \right).
\end{align*}
Here, $\lambda_i$ are the Chern roots of $T_X$.

\todo{\james{What is $\gamma$?}}

\begin{conjecture}[Gamma conjecture]
    For this specific $\bQ$-structure, the $\bQ$-structure axioms and opposedness axioms for a variation of nc-Hodge structure are satisfied.
\end{conjecture}

\subsection*{B-model nc-Hodge structures}

We now explain the B-model nc-Hodge structure associated to a Landau--Ginzburg model: $Y$ is smooth and quasi-projective over $\bC$, and $w:Y \to \bA^1$ is a function such that $\Crit(f)$ is proper. 

The twisted de Rham cohomology is 
\[
    \cH = \rR \Gamma(Y, \Omega_Y^\bullet[u], ud - (dw \wedge-))
\]
It is a difficult theorem of Sabbah et. al. that this is a $\bZ/2$-graded vector bundle over $\bA^1$. The connection is the Gauss--Manin connection
\[
    \nabla = \partial_u - \frac{w}{u^2}.
\]
Finally, the $\bQ$-structure is given by \emph{rapid decay cohomology}: 
\[
    E_{\theta} = \rH^*(Y, w^{-1}(Y_{\theta})).
\]
Here, $\theta$ is an angle, and $w^{-1}(Y_{\theta})$ is a fiber over a point of angle $\theta$, sufficiently far from the origin. 

\begin{conjecture}
    With the data above, one has a nc-Hodge structure. 
\end{conjecture}

The main difficulty seems to be opposedness.

\subsection*{Hodge-theoretic mirror symmetry}
Mirror symmetry is usually thought of as a duality between Calabi--Yau varieties of the same dimension:
\[
    X \iff \check X
\]
Nowadays, we can take $X$ to be either Calabi--Yau or Fano, and the mirror to be a Landau--Ginzburg model $(Y, w:Y \to \bC)$.
Using nc-Hodge structures, one obtains a precise version of mirror symmetry:

\begin{conjecture}[Hodge-theoretic mirror symmetry]
    The A-model nc-Hodge structure for $X$ is isomorphic to the B-model nc-Hodge structure for $(Y, w)$.
\end{conjecture}

The fact that either side is an nc-Hodge structure in the first place is conjectural, as indicated above.

\begin{remark}
    Under the conjecture, the critical values $\Crit(f)$ correspond to the eigenvalues of the quantum operator $K$. 
\end{remark}


A different version of mirror symmetry is the homological mirror symmetry conjecture of Kontsevich, which predicts that there is an equivalence of categories
\[
    \Db(X) \simeq \mathrm{FS}(Y, w).
\]
The right-hand side has a \emph{natural} semiorthogonal decomposition up to mutation, which suggests that the left-hand side should also have a \emph{natural} semiorthogonal decomposition up to mutation.


The main takeaway is the following series of implications:
\[
    \begin{tikzcd}
        \textrm{Decomposition of B-model nc-Hodge structure} \ar[d, squiggly] \\
        \textrm{Decomposition of A-model nc-Hodge structure} \ar[d, squiggly] \\
        \textrm{Decomposition of classical Hodge structures of } X
    \end{tikzcd}
\]
You might say this is on very shaky ground, since it is built on a web of difficult conjectures. To return to firmer (and more scientific) ground, we discard the categorical point of view and working directly with differential equations. 



\subsection{Analytic decomposition via the Fourier transform}


We have the following diagram:
\[
    \begin{tikzcd}
        (1) \ar[r, "\textrm{Fourier}"] \ar[d, "\textrm{irr RH}"] &  (2) \ar[d, "RH"] \\
        (3) \ar[r, "\textrm{top. RH}"] & (4) .
    \end{tikzcd}
\]
where:
\begin{enumerate} [label = (\arabic*)]
    \item de Rham data: An algebraic vector bundle $\cH$ with connection $\nabla$ on $\bG_m$, singularities of exponential type at $0$, regular singularities at $\infty$.
    \item regular holonomic D-module on $\bA^1$ with regular sing, and vanishing de Rham cohomology.
    \item $\bQ$-Stokes structure $(I, I_S)$ of exponential type.
    \item Constructible sheaf $F$ of $\bQ$-vector spaces on $\bA^1$ such that $\rR \Gamma(F) = 0$.
\end{enumerate}
Comments:
\begin{itemize}
    \item Having exponential type at $0$, reg. sing. at $\infty$ is equivalent to having regular singularities after Fourier transform.
    \item The condition that (2) has vanishing de Rham cohomology implies that the corresponding perverse sheaf is concentrated in a single degree.
\end{itemize}

\begin{definition}
\label{def:vanishing_cycle_decomposition}
    A \emph{vanishing cycle decomposition} of $F$ is a collection of finite-dimensional $\bQ$-vector spaces $U_1, \dots, U_n$, and a collection of maps $T_{i j}:U_i \to U_j$. Here, $U_i = F_{c_0}/F_{c_i}$, where we have chosen some points $c_0, c_1, \dots, c_n$.
\end{definition}



\todo{\james{Incomplete. This is supposed to be the generalization of the classical picture of a perverse sheaf on a disc.}}

\begin{theorem}[Analytic decomposition theorem, KKP]
\label{thm:analytic_decomposition_theorem}
    An nc-Hodge structure of exponential type is equivalent to the following data:
    \begin{itemize}
        \item a finite set $\{c_1, \dots, c_n\} \subset \bC$
        \item a collection of nc-Hodge structures $(H_i, E_B, \mathrm{iso})$, $i = 1, \dots, n$
        \item Gluing data and maps $T_{ij}: (E_{B, j})_{c_o} \to (E_{B, i})_{c_0}$.
    \end{itemize}
\end{theorem}

% section nc_hodge_structures_and_mirror_symmetry (end)


\section{F-bundles} % (fold)
\label{sec:f_bundles}

From here on out, there will be no more conjectures---only theorems. The idea here is to ignore both $\bQ$-structure and issues of convergence. It turns out that the trick for this is to work over a non-archimedean field. 

We begin with a smooth rigid $k$-analytic variety, where $k$ is an algebraically closed non-archimedean field. Let $\bD$ be the germ of $0 \in \bA^1_u$.

\begin{definition}
\label{def:f_bundle}
    An \emph{F-bundle} $(H, \nabla)/B$ consists of a vector bundle $H$ over $B \times \bD$ with a meromorphic flat connection $\nabla$, such that:
    \begin{enumerate} [label = (\arabic*)]
         \item $\nabla_{\partial_u}$ has a pole of order $\leq 2$ along $u = 0$,
         \item For any tangent vector field $\xi$ in $B$, $\nabla_{\xi}$ has a pole of order $\leq 1$ along $u = 0$.
     \end{enumerate} 
\end{definition}

\begin{example}
Here is an example:
    \begin{equation*}
        \nabla_{\partial_{t_i}} = \frac{1}{u} \begin{pmatrix}
            0 \\
            & \ddots \\
            & & 1 \\
            & & & \ddots \\
            & & & & 1
        \end{pmatrix}, \quad \nabla_{\partial_u} = \partial_u - \frac{1}{u^2} \begin{pmatrix}
            t_1 \\
            & \ddots \\
            & & t_n
        \end{pmatrix}
    \end{equation*}
\end{example}

For any $b \in B$, we obtain a map
\begin{equation*}
    \mu_b:T_b B \to \End(H_{b, 0}), \quad v \mapsto \nabla_{uv}|_{H_{b, 0}}
\end{equation*}
Flat implies that the image consists of commuting operators. 

\begin{definition}
\label{def:overmaximal}
    An F-bundle $(H, D)$ is \emph{overmaximal} (resp., \emph{maximal}) at $b \in B$ if there exists $h \in H_{b, 0}$ such that 
    \[
        T_b B \to H_{b, 0}, \quad v \mapsto \mu_b(v)h
    \]
    is an epimorphisms (resp., isomorphism).
\end{definition}

If the F-bundle is maximal, then there is a commutative product on the tangent bundle $TB$ given by
\[
    \mu_b(v_1 \star v_2) = \mu_b(v_2) \circ \mu_b(v_1)(h) \quad \textrm{quantum product in A-model}
\]
We also obtain an Euler vector field $\euler$ on $B$, such that 
\[
    \mu_b(\euler) = K_b = \nabla_{u^2 \partial_u}|_{u = 0}.
\]
Here, $K_b$ is the same $K$ as in quantum cohomology.

\begin{example}
\begin{align*}
    \partial_{t_i} \star \partial_{t_j} &= \delta_{ij} \partial_{t_i} \\
    \euler &= \sum_j t_i \partial_{t_i}.
\end{align*}
\end{example}

\subsection{Nonarchimedean decomposition theorem}

Let $(H, \nabla)/B$ be a maximal F-bundle, and let $b \in B$ be a point. At $b$, there is a generalized eigenspace decomposition
\begin{equation}
\label{eq:eigenspace}
    H_{b, 0} = \bigoplus E_i.
\end{equation}

\begin{theorem}[Nonarchimedean decomposition theorem]
\label{thm:nonarch_dec_thm}
    Then $(H, \nabla)/B$ locally splits into a product of maximal F-bundles $(H_i, \nabla)/B_i$, extending \eqref{eq:eigenspace}.
\end{theorem}



\begin{remark}[Comparison]
    It is possible to make a comparison with the vanishing cycle decomposition for nc-Hodge structures. One can show that there is a choice of paths from $c_i$ to $c_0$, such that the associated vanishing cycle decomposition 
\end{remark}


% section f_bundles (end)


\section{Blowup decomposition} % (fold)
\label{sec:blowup_decomposition}

Let $X$ be smooth, projective over $\bC$, $Z \subset X$ a smooth subvariety of pure codimension $r \geq 2$, and $\tilde X \to X$ the blowup of $X$ along $Z$. Recall that we have a decomposition of cohomology:
\[
    \rH^*(\tilde X) \simeq \rH^*(X) \oplus \bigoplus_{i = 1}^r \rH^{*}(Z)[-2i]
\]
One goal might be to extend this to a decomposition of nc-Hodge structures. But this is too conjectural, so we instead try to extend it to a decomposition of F-bundles. 

First, we need to care of curve classes, since these are part of the coefficients of F-bundles. To simplify notation, let

\subsection*{Iritani's theorem}

\begin{align*}
    \bQ[Q] &= \bQ[[\mathrm{NE}(X, Z)]] = \bQ[[Q^d, d \in \mathrm{NE}(X, Z)]] \\
    \bQ[\cQ] &= \bQ[[Q^d, xy^{-1}, Q^{\phi_* \tilde d}y^{-[E]\tilde d}:d \in \mathrm{NE}(X, \bZ), \tilde d \in \mathrm{NE}(\tilde X, \bZ)]] \\
    \bQ[[\tilde Q]] &= \bQ[[\mathrm{NE}(\tilde X, \bZ)]] \\
    \bQ[[Q_Z]] &= \bQ[[\mathrm{NE}(Z, \bZ)]]
\end{align*}

Here $\mathrm{NE}(X, Z)$ are the effective curve classes supported on $Z$, etc.

From embedding everything into $\mathrm{Bl}_{Z \times 0}(X \times \bP^1)$, we get embeddings:
\begin{align*}
    \bQ[[Q]] &\to \bQ((q^{-1/5}))[[\cQ]] \quad s = \begin{cases}
        r - 1 & r \equiv 0 \mod 2\\
        2(r - 1) & r \equiv 1 \mod 2
    \end{cases} \\
    \bQ[[\tilde Q]] &\to \bQ((q^{-1/5}))[\cQ] & \tilde Q^{\tilde d} \mapsto Q^{\phi_* d} q^{-[E] \tilde d} \\
    \bQ[[Q_Z]] &\to \bQ((q^{-1/5}))[\cQ], \quad Q^d_Z \mapsto Q^d_Z \mapsto Q^{\varphi_*d} q^{-c_1(N_{Z/X})d/(r - 1)}
\end{align*}

\begin{theorem}[Iritani]
\label{thm:iritani}
    After pullback to $\bC((q^{-1/5}))[[\cQ]]$, there exists a formal invertible change of variavles
    \[
        \rH^*(\tilde X) \mapsto \rH^*(X) \oplus \rH^*(Z)^{\oplus r - 1}, \quad \tau \mapsto (\tau(\tilde \tau)), \{\zeta_j \cdot (\tilde \tau)\}_{0 \leq j \leq r - 2}
    \]
    defined over $\bC((q^{-1/5}))[[\cQ]]$, and an isomorphism of formal F-bundles
    \[
        (H_{\tilde X}, \nabla_{\tilde X}) \simeq \tau^*(H_X, \nabla_X) \oplus_{j = 0}^{r - 2} \zeta^*_j(H_Z, \nabla_Z)
    \]
\end{theorem}

This is a really interesting theorem, and is a ``packaged'' way of answering the question: How do the Gromov--Witten invariants of $X$ change under a blowup?

\subsection*{Non-archimedean F-bundles}

Consider $(\tilde H, \tilde \nabla)/\tilde B^{\max}$ for $\tilde X$, $(H', \nabla')_{B^{',\max}}$. Here,
\[
    X' = X \sqcup_{r - 1} \bigsqcup Z.
\]


\begin{theorem}
There exists unique isomorphism of maximal F-bundles between $(\tilde X, \tilde \nabla)$ and $(H', \nabla')$ over an analytic domain $(\tilde U)$ in $\tilde B^{\max}$, and the analytic domain $U'$ in $B^{',\max}$. The union of different choices of $\tilde U$ is connected and nonempty; same for $U'$.
\end{theorem}

\begin{remark}[$B^{\max}$]
    When $X$ is smooth, projective, we had a formal base $B_{\mathrm{formal}}$ for the variation of nc-Hodge structure. For $B^{\max}$, we first consider let $k$ be a non-archimedean base field. Then we consider $\rH^2(X, k^*)$, which has a valuation map to $\rH^2(X, \bR)$ (taking the valuation of the coefficient). In $\rH^2(X, \bR)$, one has the ample cone. The preimage in $\rH^2(X, k^*)$ is denoted $B^2$. Then 
    \[
    B^{\max} = B^2 \times (\textrm{open unit disc in all } t_i : \deg t_i \neq 2  ).
    \]
\end{remark}

% section blowup_decomposition (end)

\section{Atoms} % (fold)
\label{sec:atoms}

Here, the goal is to relate F-bundles with motives. We will work with three different fields. First, $K$ is any field; we will work with varieties over $K$. Then we consider $k$, which has characteristic $0$; it will be the coefficient field of the cohomology theory. Finally, we will consider an algebraically closed nonarchimedean field, $\bk$. \james{Previously, this was denoted $k$. Sorry!}

Let $\cC$ be a semisimple neutral Tannakian category over $k$, so that $\cC = \mathrm{Rep}(G)$, for $G$ a pro-reductive group. Assume that $G$ has a central element $\epsilon \in G$ of order $2$. Let $\rH^*$ be a Weil cohomology theory of projective $K$-varieties, taking values in $\cC$, satisfying a \emph{Mumford--Tate normalization} condition: 
\begin{enumerate} [label = (\arabic*)]
    \item If $\rH^2(\bP^1)$ is a trivial rank $1$ $G$-module. 
    \item For any smooth, projective $K$-variety $X$ and any $i \in \bZ$, $\epsilon$ acts on $\rH^i(X)$ by $(-1)^i$.
\end{enumerate}

\begin{example}
    Let $k = \bQ$, $\cC_0$ be the category of pure, polarizable $\bQ$-Hodge structures. Then $\cC_0 = \mathrm{Rep}(G_0)$, where $G_0$ is the Mumford--Tate group. There is a natural homomorphism from $G_0$ to $\bG_m$ corresponding to the action of $G_0$ on $\rH^2(\bP^1)$; the kernel $G$ satisfies the conditions above. Then $\epsilon$ comes from the Deligne torus. 
\end{example}

\begin{example}
    One can take $C = \mathrm{Rep}(G)$ to be Andre's category of motivated cycles (so $\rH^*(X)^G$ is the subgroup spanned by motivated cycles).
\end{example}

\subsection*{The proof}

\begin{definition}
Let $B$ be the germ of a smooth $\bk$-analytic space at a rigid point (a point corresponding to a maximal ideal). A $G$-equivariant maximal $F$-bundle is called a $G$-atom if the action by the Euler field $\euler$ has a single eigenvalue. ($\euler$ is the residue of the second order pole of the connection.)

Two $G$-atoms are \emph{equivalent} if they come from a $G$-equivariant $F$-bundle over a connected base (a connected smooth $\bk$-analytic space).
\end{definition}

For any $G$-equivariant maximal F-bundle $(H, \nabla)/B$, then we consider the locus $B_0 \subset B^G$ where the number of distinct eigenvalues of $\euler$ is maximal. Finally, we apply Theorem~\ref{thm:nonarch_dec_thm}: Consider the finite \'etale covering $\bB \to B_0$. Then each connected component of $\bB$ gives an equivalence class of $G$-atom. 

We can look at a coarser invariant. Each $G$-atom gives an isomorphism class of finite-dimensional $G$-representation. 

Let $G$ be in examples $1$ or $2$. Then $G$ acts on non-archimedean $A$-model $F$-bundle $(H, \nabla)/B$ associated to a smooth, projective variety $X$, $\Atom(X)$. The blowup decomposition theorem says that
\[
    \Atom(\tilde X) = \Atom(X) + \sum_{r - 1} \Atom(Z).
\]
This is a sum of multisets. 

\begin{theorem}
\label{thm:main_theorem}
    If $K_X$ is nef, then $\Atom(X)$ is a singleton. 
\end{theorem}

Consider a general cubic $4$-fold. The eigenvalues of $\euler$ at a specific point are given by:
\begin{equation*}
    \begin{array}{ccc}
        & 1 & \\
        & 1 & \\
        & 1 & \\
        & 1 \\
        & 1
    \end{array} + \begin{array}{ccc}
        \\
        \\
        1 & 22 & 1 \\
        \\
        &
    \end{array}
\end{equation*}
$\dim V = 24$, $\dim V^G = 2$. This is computed at a specific point. 

When we go to a general point, $V$ may split further into $V' \oplus \cdots$. But
\[
    \dim V'^{p - q = 2} \geq 1, \quad \dim V'^G \leq 2.
\]
But the claim is that such a $V'$ does not appear in the F-bundle of any variety $S$ with $\dim S \leq 2$. This is because for surfaces with $h^{2,0} \neq 0$, there is a birational model with $K_S$ nef. Then $\Atom(S)$ is a singleton, and $\dim V'^G \geq 3$.



% section atoms (end)

















%%%%%%%%%%%%%%%%%%%%%%%%%%%%%%%%%%%%%%%%%%%%%%%%%%%%%%

\newpage
\addtocontents{toc}{\vspace{\normalbaselineskip}}
\bibliographystyle{amsalpha}
\bibliography{tony}

%%%%%%%%%%%%%%%%%%%%%%%%%%%%%%%%%%%%%%%%%%%%%%%%%%%%%%

\end{document}

%%%%%%%%%%%%%%%%%%%%%%%%%%%%%%%%%%%%%%%%%%%%%%%%%%%%%%

